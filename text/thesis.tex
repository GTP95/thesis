\documentclass{report}

\usepackage{biblatex}
\usepackage{dsfont}
\usepackage{mathtools}
\usepackage[colorinlistoftodos]{todonotes}

\addbibresource{bibliography.bib}

\begin{document}

\title{Designing and implementing a participant login solution for PEP using IRMA}
\author{Giacomo Tommaso Petrucci}
\maketitle

\begin{abstract}
	Polymorphic Encryption and Pseudonimisation (PEP) is a project developed at iHub to let researcher collect medical data of people taking part in studies, while preserving their 
	privacy. To do so, it uses ElGamal cipher's properties that allow to re-key and re-shuffle encrypted data. While this system effectively safeguards a participant's privacy, it 
	also makes it non-trivial to design a way for the participants to access their own data: due to PEP's privacy goals, the typical login with username and password is out of the
	question. 
\end{abstract}

\section{Introduction}
Polymorphic Encryption and Pseudonimisation (PEP) is a secure data repository with the goal of storing privacy-sensitive data while preserving the identity of the data owners as
much as possible. From a high-level perspective, it looks like a traditional database, because it is structured as a two-dimensional table. But it is not a database, as it
uses polymorphic encryption to protect the privacy and identity of the data owners, and it has the concept of data cards \cite{pep-blueprint}.\todo{Here I'm citing PEP's blueprint
document, but it isn't public yet. Can I still cite it?}\par
To store data in encrypted form, PEP uses the ElGamal cipher. Thanks to ElGamal's re-key operation, it is possible to store the data by an untrusted party even before knowing who
will need to get access to the data. Then the data can be subsequently re-keyed to grant access to the intended person or entity, without exposing the plaintext to the storage provider.
An entity might need a global identifier for the people whose data is stored inside PEP, or to store an already existing identifier but without accessing the original identifier. To
solve this problem, PEP uses ElGamal's re-shuffle operation, that lets derive many identifiers from a "base identifier" without disclosing that base identifier \cite{peppaper}.\par
To enable reproducibility of historical queries, PEP has the concept of data cards. Each piece of data is stored inside a data card which is in turn stored in one of the table's
cells. If the data contained inside a cell needs to be updated, instead of deleting the already existing card, a new data card is generated and stored "on top" of the previous data
card. Thus, each cell contains a deck of cards, and it is possible to see which data a query would have returned in a specific past point in time. \par
PEP's current focus is healthcare, but its general architecture can be used for different purposes. NOLAI \cite{nolai} \cite{pepproject} recently announced that it will use PEP to
collect their research data. \par
A side effect of PEP's privacy focus is that it didn't ship with a way for study participants, or people whose data is stored inside PEP in general, to get access to their own data.
This work is aimed at finding and developing a solution to this issue.

\section{An introduction to PEP's internals}
At the heart of PEP's design, are ElGamal's re-randomize, re-key and re-shuffle operations. To this cryptographic foundation, other elements are added to get a complete functional system. In
particular, PEP uses a form of role-based access control (RBAC)\cite{rbac} to manage users' privileges. It also employs hybrid cryptography for performance reasons and a
distributed architecture to minimize the trust needed in each component. This section will first present a quick recap of ElGamal cipher and in particular of the re-randomize, re-key and re-shuffle
operations. Then it will briefly discuss PEP's design and focus on PEP's way to manage users' privileges.

\subsection{ElGamal recap}
ElGamal is a public key cryptosystem based on the hardness of computing discrete logarithms \cite{elgamal}. It was originally devised to work on Galois fields, but it has since
been adapted to work on elliptic curves \cite{elliptic-elgamal}. As PEP uses the latter version, this is the one that will be presented here.\\
Assume Alice would like to send an encrypted message to Bob. First, Bob has to choose the domain parameters: a cyclic group composed by the points on an elliptic curve modulo a
prime number, called $E(\mathds{F}_p)$, and a generator for such a group, called $G$. He then generates a keypair by generating a uniformly random $b \in \mathds{F}^*_q$ and computing
$B=[b]G$. $b$ is Bob's private key, and $B$ is the public key.\\
Now Alice needs to encode her message $m$ to $M \in E(\mathds{F}_q)$, i.e. a point on the curve. She can then proceed to encrypt it by generating a uniformly random nonce $a \in
\mathds{F}^*_q$ and computing

$$A=[a]G$$
$$C=M+[a]B$$

She then sends the pair $(A, C)$ to Bob. \\
Bob decrypts the message by computing $M=C-[b]A$ and then decoding $M$ to $m$.

\subsubsection{Homomorphic property}
ElGamal over elliptic curves is homomorphic with regards to addition. This is the basis for the re-randomize, re-key and re-shuffle operations used inside PEP \cite{peppaper}.
\newline \newline
Given a ciphertext $(A, C)$ where, as before, $A=[a]G$ and $C=M+[a]B$.
\textbf{Re-randomization:} it is possible to derive a new ElGamal ciphertext that decrypts to the same plaintext as the original ciphertext, but is different from the original
ciphertext. To obtain this, we pick a random $r \in \mathds{F}^*_q$ and compute $A'=[r]G+A$, $C'=[r]B+C$. The pair $(A', C')$ is the re-randomization of the pair $(A, C)$. \newline
\textbf{Re-keying:} it is possible to obtain a new ElGamal ciphertext that decrypts to the same plaintext as the original ciphertext, but under a different private key. Assuming
that we would like to obtain a new ciphertext encrypted using the public key $[k]B$ from a ciphertext that has been obtained using the public key $B$. We proceed by first
finding $k^{-1} \in \mathds{F}_q$. Then we compute $A'=[k^{-1}]A$. The pair $(A', C)$ decrypts to the same plaintext as the pair $(A, C)$, but under a different private key $kb$.
\newline
\textbf{Re-shuffling:} it is possible to transform a ciphertext in a way that it decrypts to a re-shuffled version of the message M, i.e. it decrypts to $nM$ for some $n \in
\mathds{F}_q$. To do this, we compute $A'=nA$ and $C'=nC$. The pair $(A', C')$ decrypts to $nM$. \newline
For proofs of these properties, see \cite{pep-whitepaper}. 

\subsection{Data structures}
The main data structure used by PEP is a bidimensional table divided into rows and columns \cite{pep-blueprint}. Each row represents a study participant and each column either the 
results of some medical test or some data used to support PEP's functionalities. The cells pertaining to exam results contain a deck of data cards. \par
Each data card contains the actual results and some metadata. This part of the data card is encrypted using AES-256 \cite{AES-standard} in GCM mode \cite{GCM}. There is then an 
additional field, containing the AES key encrypted using ElGamal over Curve25519 \cite{elliptic-elgamal}. Using hybrid Cryptography enables PEP to use ElGamal's properties at a 
performance cost close to the one of using AES-256-GCM. \par
If the data contained inside a data card needs to be updated, instead of deleting the card a new one is added "on top" of it, overlaying the old data. Subsequent queries will
return the new data, unless a point in time is specified. In this case, the returned data is the data recorded in the data card that was on top of the deck at that specified
time.

\subsection{Pseudonyms}
As part of the registration procedure, each participant get assigned a participant identifier. This identifier isn't used directly, but instead it serves as a basis to generate
local pseudonyms. These pseudonyms do not depend solely on the participant identifier, but also on the usage context: different studies involving the same participants or different
researchers accessing the same dataset will get different local pseudonyms. These local pseudonyms are obtained by encrypting the participant's identifier using ElGamal, and then
applying the re-shuffle operation \cite{pep-blueprint}. \par
Sometimes it is necessary to generate an identifier to link the result of a specific test, e.g. a blood test, to a participant. To do this, PEP is able to generate short
pseudonyms. These short pseudonyms aren't derived from the participant identifier, and are instead generated as a random number concatenated with a checksum to prevent
transcription errors. These short pseudonyms are then recorded inside a specific column related to the specific exam they are used for \cite{pep-blueprint}.

\subsection{Data access}
This section mentions PEP users. These are in principle (and, before this thesis work gets implemented, also in  practice) different from study participants. A PEP user is someone
that interacts with PEP, while a study participant is someone taking part in a study. \par
To be able to interact with PEP's table, a user needs a ticket granting the correct permissions for the required operation. PEP implements a way to specify access rules for
specific user groups, and then a specific user is assigned to a group. These rules are then used to determine whether to grant a ticket for a specific operation or not. This
constitutes a form of role-based access control \cite{rbac}. To be able to bootstrap this permissions system, it is needed that someone has the right of writing rules for access
groups, to specify access groups and to add users to groups. For this reason, PEP has two builtin access groups: "Data Administrator" and "Access Administrator" \cite{pep-blueprint}. A
member of the former group is called a data administrator, and has the permissions needed to modify columns, column groups, and participant groups. A member of the latter group is
called an access administrator, and has the right to modify the access rules used to issue tickets, modify and read the user list to issue authentication tokens, and manually issue
authentication tokens. These two roles are given to distinct people, so that they would need to collude to access a specific piece of data.\todo{Think a bit about this}

\subsection{authentication}



\iffalse
\section{How to give access to study participants}
\subsection{Some initial ideas}
The first idea to let users download their data from PEP was to use IRMA to disclose the data needed to generate the pseudonym to some component that then proceeds to generate it.
This doesn't work because the pseudonym is generated by encrypting a random string. But this string gets stored inside Salesforce, so the next idea was to interface with
Salesforce. This could have posed some issues though:
\begin{enumerate}
		\item Is the seed stored along some data that uniquely identifies the user?
		\item Does Salesforce have some API to interface with other software?
\end{enumerate}
Let's analyze these issues. If the seed is stored alongside personal data, then getting it from Salesforce could expose personal identifiable information. How? One could argue that
they shouldn't store anything inside Salesforce anyway... But it is done to be able to de-anonymize users in special circumstances. And here we get to the point: better not touch
Salesforce, unless those special circumstances arise. The second issue is about API access. Salesforce does have it, but only for certain editions \cite{salesforce}. The PEP team
blackballed the idea of accessing Salesforce. The next proposal was to give participants their own ID (main pseudonym) during the enrollment procedure as an IRMA credential, but
providing them with this ID could disclose more information than necessary.
Since this first idea was discarded, I considered the following approaches:
\begin{enumerate}
		\item Give participants their own ID during enrollment phase.
		\item Give participants a re-shuffled ID.
		\item Create a fake study in which the participant has the researcher role.
		\item Use IRMA’s chained session: it is possible to derive a card from other cards the user already has by having the user disclose he attributes contained there. The server can then apply some function on the attributes and derive a new card from that.
\end{enumerate}

The first approach could pose unnecessary risks. The participant's pseudonym is used to obtain the re-shuffled pseudnyms that identify the participant in different studies and in
different datasets. To be able to derive these pseudonyms, an attacker would also need to get the re-shuffle key, but not leaking the pseudonym outside of PEP in the first place would still be
better. \\
The second approach would solve the problem of leaking the participant's pseudonym, but participants would still need to be provided with some key material to ab able to decrypt
the data once downloaded. [Is this true though? Couldn't this just work the same as I have it implemented now?]. \\
In a similar fashion to the previous approach, it would be possible to create a fake study inside PEP with a single participant, one for each user. Then the users would be given
the researcher role inside their own fake study to be able to access their own data. Also this option would require to provide them with some key material, opening the door to keym
management issues [again, is this actually needed? See also previous note]. \\
The last approach is a way to sidestep the issue of providing study participants with their own pseudonym or a re-shuffled pseudonym. But it would require some way of linking the
IRMA identity to the participant.\\

\subsection{The chosen solution}
PEP uses rbac, let's use that too! First, create a column group containing all the columns pertaining to exam results. Then, for each participant, create a new participant group
containing only that participant. Create a new user and apply permissions as needed. And SBAM! You've got it! [Insert here actual participant enrollment procedure and provide
details].
\subsection{The pain of implementing the solution}

\fi

\printbibliography
\end{document}
